\documentclass{article}
\usepackage{amsmath}
\usepackage{graphicx}
\usepackage[T2A]{fontenc}
\usepackage[utf8]{inputenc}
\usepackage[russian]{babel}
\begin{document}





\section{Задача 1.5.8}

\begin{align*}
    \lambda & = 34 \, \text{см} = 34 \cdot 10^{-2} \, \text{м} \\
    m & = 27.6 \cdot 10^{-3} \, \text{кг} \\
    A & = 12 \cdot 10^{-6} \, \text{см}^2 \\
    I_0 & = \frac{m}{12} = \frac{27.6 \cdot 10^{-3}}{(34 \cdot 10^{-2})^2} \approx 2.38 \cdot 10^{-3} \, \text{кг} \cdot \text{м}^2 \\
    T_{\text{min}} & = 2 \pi \sqrt{2I_0 / m} \, \text{с} \\
    T_{\text{min}} & = 2 \cdot 3.14 \cdot \frac{\sqrt{2 \cdot 2.38 \cdot 10^{-3}}}{27.6 \cdot 10^{-3}} \approx 0.089 \, \text{с} \\
    a^* & = \sqrt{\frac{I_0}{m}} = \sqrt{\frac{2.38 \cdot 10^{-3}}{27.6 \cdot 10^{-3}}} \approx 0.098 \\
    t_{\text{cp}} & = 0.925 \\
    \text{Спр. } T_{\text{min}} & = 0.025 - 0.089 \, \text{с}, \quad 10\% = 90.3\%
\end{align*}

\section{Вывод}
Из данной лабораторной работы я ознакомилась с методами и применением закона сохранения физических величин при малой условной и пульсации. Я также определила ювенильные силы тяжести и мощность различных маятников.

\end{document}
